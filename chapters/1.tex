\chapter{Introduzione}
\label{cap1}

Lo spettro di servizi di reti sociali (\textit{social network services}, SNS) presenti è ampio e variegato per dominio d'interesse, contenuti, tipologia di utenti e servizi offerti. Molte persone dispongono di un'eterogeneità di profili su diversi social network services per soddisfare i loro molteplici bisogni. Le informazioni presenti su ognuno di questi servizi possono differire, e quindi le informazioni riguardanti un individuo si ritrovano ad essere sparse su servizi diversi e spesso non facilmente riconducibili ad una singola fonte. Per poter integrare queste informazioni è necessario sapere identificare un individuo attraverso diversi social network services. La difficoltà riscontrata è dovuta in parte alla natura del web e all'azione della maggior parte dei SNS\footnote{Social Network Services} che preservano l'anonimità dei loro utenti, permettendogli di scegliere liberamente uno username, uno pseudonimo, invece che la loro reale identità. A questo si unisce il fenomeno di utilizzare diversi username e sistemi di autenticazione sui diversi servizi che si utilizzano. Inoltre, nonostante si riscontri un crescente numero di servizi e utenti\cite{openidtrend} che adottano tecnologie Single-Sign-On (SSO) come openID\footnote{http://openid.net}, dove gli utenti possono autenticarsi utilizzando un singolo username ( ad esempio, gli utenti possono autenticarsi su Google+ e YouTube utilizzando il loro account GMail), questi non coprono la totalità dei servizi e degli utenti.\newline Vi sono diverse applicazioni che beneficerebbero di un sistema di profile matching. Una applicazione è riscontrabile nell'area di marketing e business, dove una maggiore profondità di percezione e comprensione dei propri clienti, apporterebbe una migliore comunicazione e una più significativa espressione di messaggi con le persone. Strumenti di audience analysis come quello sviluppato recentemente da Facebook\cite{fbinsights} permettono a venditori di apprendere maggiori informazioni riguardo il loro target, includendo informazioni aggregate riguardanti disposizione geografiche, demografiche, comportamento negli acquisti e altro. Questi servizi trarrebbero beneficio collegando gli individui tra comunità favorendo l'integrazione delle informazioni presenti sui diversi SNS. Una seconda applicazione, come alcuni studi hanno sottolineato\cite{zafarani13}, propone questo metodo come soluzione alternativa al problema di age verification. Se venissero fornite informazioni fittizie (ad esempio un'età incorretta), emergerebbero delle inconsistenze di informazione tra diversi SNS. Individuare queste inconsistenze può aiutare a fornire un primo approccio verso la risoluzione al problema di age verification. Una prima via per individuare queste inconsistenze è connettere le diverse identità di un utente attraverso diversi social network services. Per affrontare il problema di identificazione si potrebbero utilizzare una composizione di molteplici informazioni reperibili sui diversi SNS. Tendenzialmente si potrebbero usare tutti i dati a disposizione, che spaziano da quelli di profilo ad esempio username e avatar, dati personali anagrafici come nome, cognome, luogo e data di nascita o le connessioni con altri profili del SNS. Diversi di questi dati però potrebbero non essere disponibili sui vari servizi. Alcuni studi\cite{perito11}\cite{zafarani13} hanno approcciato questo problema prendendo in considerazione unicamente gli username. In termini di disponibilità delle informazioni, gli username sembrerebbero il denominatore comune disponibile su tutti i SNS. Solitamente uno username è formato da una serie di caratteri alfanumerici o da indirizzi email, grazie ai quali gli utenti riescono ad accedere al servizio. Questi sono tipicamente unici per ognuno dei SNS, mentre la maggior parte delle informazioni personali, anche la combinazione di nome e cognome non lo sono. Formalizzeremo il problema usando gli username come entità disponibile su tutti i servizi. Considerando gli username, ci sono due problemi che devono essere risolti per identificare un utente:

\begin{enumerate}
	\item Dati due username, \textit{u\textsubscript{1}} e \textit{u\textsubscript{2}}, riuscire a determinare se appartengono allo stesso individuo
	\item Dato uno username \textit{u} di un individuo \textit{I}, trovare gli altri username di \textit{I}
\end{enumerate}

Il primo punto può trovare risposta in due passi: 1) troviamo l'insieme di tutti gli username \textit{C} che probabilmente appartengono all'individuo \textit{I}. Denotiamo l'insieme \textit{C} come username candidate e, 2) per tutti gli username candidate \textit{c} $\in$ \textit{C}, controlliamo se \textit{c} e \textit{u} appartengono allo stesso individuo. Quindi, se gli username candidate \textit{C} sono noti, il problema 2 si riduce al problema 1. Possiamo rispondere al problema 1 trovando una funzione di identificazione \textit{f}(\textit{u,c})

\[ f(u,c) = \left\{
\begin{array}{l l}
	1 & \quad \text{se \textit{c} e \textit{u} appartengono allo stesso \textit{I} ;}\\
	0 & \quad \text{altrimenti}
	\end{array} \right.\]

possiamo assumere noto che lo username \textit{u} appartenga all'individuo \textit{I} e considerare \textit{c} il candidate username di cui vorremmo verificare l'appartenenza a \textit{I}. In altre parole, \textit{u} è l'informazione a priori (prior) fornita per \textit{I}. La funzione di identificazione può essere generalizzata assumendo che l'informazione a priori sia un insieme di username \textit{U} = \{\textit{u\textsubscript{1}},\textit{u\textsubscript{2}},...,\textit{u\textsubscript{n}}\} e la nostra funzione di identificazione \textit{f}(\textit{.,.}) diventerebbe

\[ f(U,c) = \left\{
\begin{array}{l l}
	1 & \quad \text{se \textit{c} e l'insieme \textit{U} appartengono a \textit{I} ;}\\
	0 & \quad \text{altrimenti}
	\end{array} \right.\]  dove \textit{U} è l'informazione a priori che abbiamo di un individuo \textit{I}, in questo caso un insieme di username, e \textit{c} è lo username candidato di cui vorremmo testare l'appartenenza allo stesso \textit{I}.\newline
Questa tesi pone come obiettivi la progettazione e l'implementazione di una soluzione
che consenta di stabilire la correlazione di identità tra profili di individui presenti su diversi social networks. Una prima fase del lavoro, come vedremo nel capitolo 2, consisterà nella implementazione di un algoritmo che conduca alla risoluzione del problema di profile matching individuando una serie di modelli comportamentali mostrati dagli utenti nello scegliere uno username che possono aiutare nell'identificazione di utenti tra differenti SNS. Implementeremo parte delle feature presentate da Zafarani et al\cite{zafarani13} che catturano questi modelli e proporremo alcune feature addizionali, per poi discutere le differenze riscontrate nell'utilizzo di queste. Questo problema verrà affrontato utilizzando un dataset acquisito, come descritto nel capitolo 3, estraendo dati da fonti pubbliche raggiungibili su \textit{internet} attraverso \textit{web scraping} o \textit{web data extraction}, una tecnica per l'estrazione di dati da pagine web. Verranno poi proposte e discusse nel capitolo 4, le soluzioni implementate per condurre l'analisi dei dati raccolti e l'implementazione di un modello predittivo, in particolare un modello di riconoscimento di pattern, capace di attuare predizioni accurate in base alle osservazioni fatte sui dati estratti descritti precedentemente.
