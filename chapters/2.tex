\chapter{Pattern comportamentali e costruzione di features}
\label{cap2}
Gli individui mostrano spesso pattern comportamentali nella scelta dei loro usernames. Questi pattern, risultanti in ridondanza di informazione, possono essere utili per identificare individui su diversi social networks.
I soggetti potrebbero evitare queste ridondanze selezionando usernames in modo che risultino completamente diversi dai loro altri usernames. In questa maniera gli usernames risulterebbero essere talmente differenti che dato uno username, nessuna informazione riguardante gli altri usernames potrebbe essere estratta.
Idealmente per raggiungere questo stato di indipendenza, l'individuo dovrebbe scegliere uno username che presenti entropia massima. Ovvero uno username composto da una lunga sequenza di caratteri, lunga quanto il massimo consentito dal sistema, senza ridondanze - una sequenza di caratteri completamente casuale.
Sfortunatamente, tutti questi requisiti non vengono incontro alle abilità umane. Gli esseri umani hanno difficoltà a memorizzare lunghe sequenze, con la possibilità della memoria a breve termine di ricordare 7$\pm$2 elementi. Queste limitazioni risultano condurre gli individui a selezionare generlamente usernames \textit{non lunghi}, \textit{non casuali} e che presentano \textit{abbondante ridondanza}.
Queste proprietà possono essere catturate adottando features specifiche.

Possiamo suddividere questi pattern comportamentali in tre categorie:
\begin{enumerate}
  \item Pattern dovuti a limitazioni umane
  \item Fattori esogeni
  \item Fattori endogeni
\end{enumerate}

Discuteremo dei comportamenti di ognuna di queste categorie elencate e delle features che possono essere estrapolate sfruttando questi pattern.


\section{Pattern dovuti a limitazioni umane}
\paragraph{Fattori esogeni}
\section{Fattori endogeni}
